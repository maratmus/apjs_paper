%\documentclass[12pt,preprint]{aastex}
\documentclass[numberedappendix,apj,twocolumn]{emulateapj}
\usepackage{graphicx}
\usepackage{lipsum}
\usepackage{subfigure}
\usepackage{amssymb,amsmath}
\usepackage{natbib}
\bibliographystyle{apj}
\usepackage{epsfig}
\setcitestyle{notesep={; }} %changes "," to ";" in Lang (2014, hereafter L14)
%\renewcommand{\thefootnote}{\fnsymbol{footnote}}
%\graphicspath{{./allfigures/}}

\begin{document}

%\title{Optical-faint, Far-infrared-bright {\textit{\textbf Herschel}} Sources
%in the CANDELS Fields: Ultra-Luminous Infrared Galaxies at \boldmath$z>1$ and
%the Effect of Source Blending
%\footnotemark[$\star$]}\footnotetext[$\star$]{Herschel is an ESA space 
%observatory with science instruments provided by European-led Principal
%Investigator consortia and with important participation from NASA.}

\title{Title
}
\author{Marat Musin \altaffilmark{1} \altaffilmark{2}, 
Haojing Yan \altaffilmark{2}, 
}

\altaffiltext{1}{Chinese Academy of Sciences South America Center for Astronomy
(CASSACA), National Astronomical Observatories, Chinese Academy of
Sciences, Beijing 100012, China}
\altaffiltext{2}{Department of Physics \& Astronomy, University of Missouri,
Columbia, MO 65211, USA}



\begin{abstract}

%\lipsum

\end{abstract}

\keywords{
 infrared: galaxies --- submillimeter: galaxies ---  galaxies: starburst ---
 methods: data analysis
}

\section{Introduction}

\subsection{Lilly-Madau formalism}


\subsection{SED fitting as a standard technique of mass and redshift estimation}
SED fitting is now a standard technique of deriving stellar mass and photometric redshifts for a large set of galaxies. In this method multi-band photometry for a given galaxy is fitted to a series of a templates predicted by a certain stellar population synthesis (SPS) model. The best-fit template gives the parameters of the galaxy, including its redshift and mass.
Historically, SPS models were using restframe optical photometry. One caveat is the degeneracy between the dust extinction and age of the stellar population, as both make the color of galaxy red, i.e. galaxy can be red because it is intrinsically red with no young massive star and ongoing star-formation, or it can be very dusty, or it can be metal-rich and metals effectively absorb light in the bluer bands. Solution to this is to implement restframe near-IR where light suffers much less extinction (comparing to restframe UV and optical) and thus the degeneracy can be broken.
We aim to build the largest sample of galaxies with optical and near-IR photometry over a large sky area. The natural choice for us then is to use optical Sloan Digital Sky Survey (SDSS) and IR all-sky data from Wide-Field Infrared Survey Explorer (WISE).


\subsection{problems associated with construction of the catalog}
Blending, poor spatial resolution in IR
our method – template fitting

\subsection{Goal of this paper}

In this paper we present our technique for construction a catalog of galaxies with reliable SED data in optical and near-IR in Stripe 82 field. We discuss data selection, sources identification in different bands and problems associated with it.

\section{Data Description}


\subsection{SDSS and Stripe 82}

The imaging component of the SDSS, which was done in five broad bands ($u' g' r' i' z'$), has covered 14,555 deg$^2$. In most area, the SDSS only scanned for one pass at an exposure time of 53.9 seconds per band, and thus is rather shallow (for example, the $r'$-band 5 $\sigma$ limiting magnitude is 22.2~mag). For this reason, in most cases the SDSS can only probe the normal galaxy population up to $z\approx 0.4$. However, the Stripe 82 region, which is a long stripe along the equator that spans $20^h < RA < 4^h$ and $-1.26^o < Dec < 1.26^o$, is the exception. It was repeatedly scanned ($\sim$ 70-90 times, depending on RA) for calibration purpose during the survey \citep{Adelman-McCarthy2007}, and thus the combined scans can reach much better sensitivities.

A number of teams have created deep Stripe 82 stacks and made them available to public. The first such stacks were produced by \citet[][]{Annis2014} based on the data obtained up to December 2005 (20-35 runs), which achieved 1-2 magnitude deeper limits than the single-pass SDSS images. Several other teams \citep[e.g.,][]{2009AJ....138..305J, 2014MNRAS.440.1296H} produced different stacks using different procedures to optimize the image qualities.

\citet[][hereafter J14]{Jiang2014} released a new version of stacks using only the images that were taken under the best weather conditions. These stacks are $\sim0.2$~mag deeper than those produced by \citet[][]{Annis2014}, reaching 5~$\sigma$ limits of 23.9, 25.1, 24.6, 24.1, 22.8~mag in $u' g' r' i' z'$, respectively, and also have better PSF characteristics. We adopt these stacks in our work.

\subsection{Structure of SDSS Stripe 82 files}

We use description from J14 to present the structure of optical data. An SDSS run (strip) consists of six parallel scanlines, identified by camera columns (Figure~\ref{fig:sdss}). The scanlines are 13.5 arcmin wide, with gaps of roughly the same width, so two interleaving strips make a stripe that consists of total 12 scanlines (columns). Their names in the direction of increasing declination are:
$col01 \rightarrow col07 \rightarrow col02 \rightarrow col08 \rightarrow col03 \rightarrow col09 \rightarrow col04 \rightarrow col10 \rightarrow col05 \rightarrow col11 \rightarrow col06 \rightarrow col12$.

%\begin{figure}[!ht]
%\includegraphics[width=6in]{THE ONE THAT HAOJING SENT ME}
%\caption{sample text}
%\label{fig:sdss_s82}
%\end{figure}

The size of each co-added SDSS image is 2854 x 2048 pixels, or roughly 18.8' x 13.5' (RA x Dec), with a pixel size of $0.396^{''}$ and an average full width at half maximum (FWHM) of $\sim1.5^{''}$ in u-band, $\sim1.3^{''}$ in g-band, and $\sim1^{''}$ in r-, i-, and z-bands. In total there are 401 SDSS images in each column and overall $12 \cdot 401 \cdot 5 = 24,060$ SDSS images in all 5 bands. Each SDSS image has a corresponding weight.fits image, that records relative weights at individual pixels.

\subsection{WISE and unWISE}

%WISE \citet{Wright2010} is a near-IR space observatory that was launched in December 2009 and mapped the entire sky with sensitivity far better than that of its predecessors, IRAS \citet{Neugebauer1984} and DIRBE \citet{Silverberg1993}. With a 0.4 m telescope on board always pointing at 90 degrees solar elongation, WISE made successful scans of the entire sky in four bands, namely w1 ($3.4 \mu m$), w2 ($4.6 \mu m$), w3 ($12 \mu m$) and w4 ($22 \mu m$). Its bands w1 and w2 are far more sensitive than w3 and w4 ( ..... mJy respectively) and can be used to extend SED of the galaxies to the near-IR, break age/dust degeneracy and account for numerous low-mass stars when calculating the stellar mass. It is natural that IR data have much worse pixel scale comparing to optical, so matching data from different data sets has always been an issue. We address this problem in the next section when talking about template fitting, but prior to that we shall have another look at WISE.

%WISE has the best resolution among near-IR telescopes (e.g. IRAS \citet{Neugebauer1984} and DIRBE \citet{Silverberg1993}) that have covered all Stripe 82. Its bands w1 (3.4 $\mu m$, 54 $\mu Jy$) and w2 (4.6 $\mu m$, 71 $\mu Jy$) are far more sensitive than  w3 (12 $\mu m$, 730 $\mu Jy$) and w4 (22 $\mu m$, 5000 $\mu Jy$) 

% and can be used to extend SED of the galaxies to the near-IR, break age/dust degeneracy and account for numerous low-mass stars when calculating the stellar mass.


WISE \citep{Wright2010} is a near-to-mid IR space telescope launched in 2009 and has performed an all-sky imaging survey in four bands at 3.4, 4.6, 12, and 22~$\mu$m (denoted as W1, W2, W3, and W4, respectively). By the end of survey operations in February 2011 it has completed three mission phases, namely WISE Cryogenic Survey, WISE 3-band Survey and NEOWISE Post-Cryo Survey \citep{2011ApJ...731...53M}.
% Complete all-sky coverage in two epochs was achieved after three mission phases, namely WISE Cryogenic Survey (120$\%$ of the sky is covered), WISE 3-band Survey (W4 band excluded, 30$\%$ of the sky is covered) and NEOWISE Post-Cryo Survey (only bands W1 and W2 were used, 70$\%$ of the sky is covered). 
%Three separate mission phases, namely WISE Cryogenic Survey, WISE 3-band Survey and NEOWISE Post-Cryo Survey allowed WISE to perform an all-sky coverage in two epochs.
%The nominal 5~$\sigma$ limits in four bands are 0.08, 0.11, 1.0, and 6.0~mJy, respectively \citep[see][for details]{Wright2010}.

%W1, W2, W3, W4 (3.4, 4.6, 12, and 22 $\mu m$, respectively). The 5$\sigma$ limits are better than 0.08, 0.11, 1, and 6 mJy for these four bands (see Wright et al. (2010) for more details).

%The AllWISE Data Release 1 was announced 2013 Nov 13, and includes data taken during the first three mission phases: the 4-band primary mission, the 3-band phase (W1, W2, W3), and the NEOWISE post-cryo phase (imaging only in W1 and W2). The data products in the AllWISE Release include coadded matched-filtered images known as the ``Atlas Images''. Atlas Images are intentionally convolved by the point-spread functions (PSFs) for better detection of isolated sources. 
%This operation reduced the resolution and created the blending problem in the crowded fields. \citet[][; hereafter D14]{Lang2014e} restored original pixel scale and preserved the spatial resolution of original images. His set of co-adds achieves 6" PSF full-width half maximum (FWHM) in W1, W2 and W3 bands and 12" in W4.

The WISE team made the AllWISE Data Release 1 in 2013 November. The image products included in this release, known as the ``Atlas Images'', are the stacks of the images from two complete sky coverage epochs taken in all three mission phases, and reach the nominal 5~$\sigma$ limits in four bands are 0.054, 0.071, 0.73, and 5.0~mJy, respectively \citep[see][for details]{2013wise.rept....1C}. To optimize the detection of isolated sources, the single-exposure images were convolved with the individual point spread function (PSF) during the stacking process. However, this operation has the drawback that it reduces the spatial resolution of the final stacks,”, which is not desirable in many applications. To deal with this problem, \citet[][hereafter L14]{Lang2014e} reprocessed all the WISE images independently without the PSF convolutions, and produced the stacks that preserve the original WISE spatial resolutions. These image products of L14,  dubbed as the ``unWISE'' images, have the PSF full-width at half maximum (FWHM) values of $6^{''}$ in W1, W2 and W3 and $12^{''}$ in W4. We use these unWISE images for this work.


\subsection{Structure of unWISE files - once again maybe I need to omit it in the paper}

The unWISE coadds are on the same tile centers as the WISE tiles with 18,240 images per band, 1.56 x 1.56 degrees each. The tiles are named by their RA, Dec center: tile "0591p530" is at RA = 59.1, Dec = +53.0 degrees; i.e., the first four digits of the tile name is $int(RA \cdot 10)$, then "p" for +Dec and "m" for -Dec, then three digits of $int(abs(Dec)\cdot 10)$. For each tile and band w1-w4, several images are produced, we shall list only the ones that we make use of:

-- unwise-0000p000-w1-img-m.fits - "Masked" image, 2048 x 2048 pixels, TAN projected at 2.75"/pixel. Background-subtracted, in units of "Vega nanomaggies" per pixel: $mag = -2.5 \cdot (log_{10}(flux) - 9)$. This is the science image, the word "masked" means that some pixels have no unmasked pixels and no measurement at all: pixel value 0 and infinite uncertainty.

-- unwise-0000p000-w1-std-m.fits - Sample standard deviation (scatter) of the individual-exposure pixels  contributing to this coadd pixel.\\

Three unWISE images centered at the same RA cover the whole width of Stripe 82 in Dec (-1.26 $\delta$ < +1.26). We shall call three such unWISE images a frame. There may be up to 72 SDSS images within one frame.

\section{Overview of Methods for Analysis}

%	SED fitting requires supplying catalogs that contains multi-wavelength photometry. Robustness of the estimated properties of the galaxies heavily depend on consistency of the photometric data. Colors (i.e. flux ratios) are of particular importance - while systematic errors in all bands leads to an error in scaling, wrong colors will result in the wrong fitted template and wrong properties assigned to the galaxy. In this section we describe the strategy for construction of such optical and IR parts of the catalog, general issues and our methods to solve it.

The most critical factor in SED fitting is consistent photometry in the involved bands, i.e., the photometry should include the same fraction of light across all bands so that the colors are defined in a consistent manner. This is challenging in our case because the spatial resolutions of WISE are at least $6\times$ worse than that of the SDSS. For this reason, the objects detected in WISE often suffer blending. Even for relatively isolated WISE sources, the photometric apertures appropriate for the (low resolution) WISE images cannot guarantee the same fraction of light being included as what is done in the (high resolution) SDSS images. Such a systematic offset, which is different for every galaxy, severely skews the SED fitting. 

To best address this problem, we opt to use the TPHOT software, which recently emerged as a robust and flexible tool to perform ``template fitting''. The basic idea is to use a high-resolution image (here an image from the SDSS) as the prior to build the morphological template of the source under question, convolve this template with the PSF of the low-resolution image (here the corresponding image from the unWISE), and fit this degraded template to the low-resolution image to obtain the total flux that is within the aperture as defined by the high-resolution image. In this way, we get reliable color information (i.e., flux ratio) in the most consistent manner. 
%It is important to note that high-resolution source does not have to be point-like - its morphological features will be preserved and fitted to the low-resolution source. This implies the biggest assumption for this technique - that morphology of the source is wavelength-independent. While generally it is not true, we anticipate that this will not create any significant bias. Firstly, because most of the galaxies have small angular sizes and such variation is negligible (galaxy at z=0.7 that is 40 kpc in diameter has an angular size of only 6 pixels), and secondly, because we chose SDSS r-band (6202.46 $\AA$), as a high-resolution image - there should not be much morphological difference between r-band and W1 or W2 bands.

While TPHOT is much more user-friendly as compared to its predecessors, running this software is still non-trivial. It not only requires careful tuning of parameters but also several tedious preparatory steps with both the high- and the low-resolution images. Here we detail our procedures.
	

\subsection{Preparing WISE and SDSS images} 

%TPHOT input files must satisfy certain requirements: the high- and the low-resolution images should have the same type of projection, reference coordinate and orientation as written in their FITS headers. We verify that SDSS and WISE images indeed have the same tangential projection and orientation ($CD1\_2 = 0, CD2\_1 = 0$). Meanwhile each WISE frame covers $\approx 3.95\, deg^{2}$ and there may be up to 72 SDSS images that cover the same area, so we use coordinates of the center of WISE images (CRVAL1 and CRVAL2) as the anchor values and run SWarp [Bertin et al., 2002] to change reference pixels in all SDSS images. The reference pixel (CRPIX1 and CRPIX2) for SDSS images can now be outside of the image itself to as far as 0.7 deg.

%TPHOT input files must satisfy certain requirements: the low-resolution background-subtracted image must have the same orientation as the high-resolution image (i.e. no rotation allowed), and the origin of one pixel must coincide. Reference pixels (CRPIX) in all SDSS images within one unWISE footprint were changed using SWarp \citep{Bertin2002} to match the world coordinate values (CRVAL) for a given unWISE image. $\approx15\%$ of the SDSS images have more than one unWISE image within its footprint. Such images were duplicated and assigned with different reference pixels, thus increasing the number of SDSS files from  4812 to 5556 images per band.

TPHOT requires that the low- and the high-resolution images have the integer multiple of their pixel scale ratio, and also have the same orientation and the same World Coordinate System (WCS) reference position, the latter of which is defined by the FITS keywords (CRVAL1, CRVAL2). We chose to change the SDSS images to match the unWISE images, because an unWISE image is always oriented to North-up and East-left and encompasses multiple SDSS footprints. 

The change to the SDSS images was done by using the SWarp software \citep[][]{Bertin2002}. The reference pixel (CRPIX1 and CRPIX2) for the SDSS images can now be outside of the image itself to as far as 0.7 deg.

$\approx15\%$ of the SDSS images lay at the interface between two adjacent WISE images. Such SDSS images were duplicated and each copy was assigned with different reference pixels, thus increasing the number of SDSS images from 4,812 to 5,556 per band.

The WISE images were rescaled from 2.75 ''/pix to 2.772 ''/pix using SWarp. The pixel scale ratio of WISE images and SDSS images is thus an integer: $2.772/0.396 = 7$.

Center pixels of saturated sources in the WISE standard deviation images have zero values and that is an invalid input for TPHOT. $\tt IRAF/imcalc$ task was used to detect such pixels and change its value to ``9999''.

%There are SDSS files that lie in the overlapping region of two adjacent unWISE images. In such a case we duplicate SDSS image and run SWarp twice, assigning new SDSS image reference coordinates from each of the adjacent unWISE images. This operation reduced the blind zone but increased the number of SDSS files from  4812 to 5556 images per band.

% * construction of the PSF and kernels
\subsection{Input SDSS source catalog} 
**Reasoning**

J14 produced object catalogs from their stacked images using $\tt SExtractor$ \citep[][]{Bertin1996}. While in general these catalogs could be used in our project, there are several caveats. The catalogs are not cleaned out from the duplicate sources in the overlap area between two adjacent images and two adjacent scanlines; catalogs are not matched between the bands and thus have different number of sources for each band; object detection threshold was set too high (DETECT$\_$THRESH = 2) and many faint objects were excluded; bright and saturated objects have multiple detections, none of which is centered on the source. For these reasons new catalogs will be constructed.

**Problem**

Due to the great depth, roughly 24.6 AB and $\approx1''$ PSF FWHM r-band was chosen as a reference band. Centroids, morphology, and other non-amplitude parameters of the detected sources are then fixed to the values from this reference band. Sensitivity, FWHM and local background varies in SDSS from band to band, so neither aperture, nor Kron flexible elliptical aperture \citep{Kron1980} can produce consistent colors in 5 optical bands, which is crucial for the subsequent SED fitting.

**General solution**

To remedy this problem, all individual images in $g'r'i'z'$ bands are convolved to the PSF of the corresponding image in the band with the worst PSF FWHM, namely u-band. For the price of losing some sources due to blending in the reference band, matched to the u-band, we extract more robust fluxes. Matching one images in one band to the PSF of the image in the different band requires knowledge of the kernel - a PSF matching function between two images. The PSF of SDSS images varies in RA (corresponds to image rows), due to atmospheric fluctuations and, as a result, different seeing. It also varied in DEC (corresponds to columns), due to camera optics and different airmass. Thus there is no universal PSF and individual PSF must be built for each SDSS image. Construction of the 27,780 PSFs in Stripe 82 (5,556 per band) is the most time-consuming part of the project.

**PSF construction**

Our strategy for the PSF construction can be described as follows. PSF is built as a combination of selected point-like sources on the image. Generally, large amount of such sources is needed to build a correct PSF that is independent of small variations in profiles of the sources \textit{I know it should be said, but I am not sure this is the best way.} The first selection of point sources that contribute to the PSF model is performed using a comparison of the core magnitude vs. total magnitude. The ratio between these two quantities represents a concentration of the source. SExtractor MAG$\_$APER and MAG$\_$BEST were used to quantify core and total magnitudes respectively. Size of the aperture depends on the particular band, but in general is $\approx$ X$''$.
The second selection involved magnitude cuts (saturated and faint sources were removed) and use of the stellarity index. The later was estimated using SExtractor CLASS$\_$STAR parameter. Sources with CLASS$\_$STAR$<$XX were rejected. The last selection only left the source in the final sample if it is not close to the edge of the image and does not have another nearby source within XX$''$. 

Finally, each SDSS image has 20 to 160 point-like sources that contribute to the PSF model. Lower value appears in a few images when there are not enough bright point sources (mostly in the less sensitive u-band), while the cut in the maximum number of the point sources was performed in order to have stable performance of the {\tt IRAF/psf} task. \textit{I know it sounds ugly, but I don't want to write the IRAF sometimes  crashes when you supply more than 160 stars.} Visual inspection is given to all constructed PSFs. If it shows some features, such as elongation, gradient of the background due to the nearby (within up to several arcmin) source, or faint non-detected blended source in the vicinity of the main profile, then manual selection of the point sources and reconstruction of the PSF is performed. {\tt IRAF/seepsf} is used to take the PSF, computed by the {\tt IRAF/psf} and build a 21x21 pixel output FITS image, consisting of the sum of the analytic function and the residual. All PSF images are subsequently normalized to the unity total counts using {\tt wcstools/sumpix} \citep{Mink1998b} and {\tt IRAF/imarith} tasks. 

**kernel construction and psf matching**

Kernels are constructed by supplying two PSF to the {\tt IRAF/lucy} task that uses algorithm developed by \citet{Richardson1972} and \citet{Lucy1974}. Matching to the PSF of the u-band is performed by {\tt IRAF/psfmatch} task that uses $g'r'i'z'$ image and a relevant kernel as an input.

**catalog construction**

Catalogs are produced running {\tt SExtractor} in the dual mode, where r-band matched to u-band ("r matched to u" for short) is used for detection and all five bands (u-band, g matched to u, r matched to u, i matched to u, z matched to u) are consequently used for photometry. Catalogs in each image are then matched by their ID.

**magnitude error correction**

Convolving the image with the PSF is often used for better detection of faint objects as it increases SNR of objects \textit{I saw it in some paper, just need to find it}. As a result, spurious detections have SNR $>$ 5 and real objects have non-realistic associated magnitude errors, which affect the choice of proper templates for SED fitting. We mitigate such problem by correcting magnitude errors for bands $g'r'i'z'$. For each image {\tt SExtractor} was ran again in the dual mode with r matched to u as a detection and original $g'r'i'z'$ SDSS images for photometry. Two catalogs for each image (original image and matched to u-band) are stacked and the mean ratio of the original SNR to matched SNR for all sources is calculated. All magnitude errors in a matched image are then multiplied by this ratio. \textit{I may insert a figure that shows this correction}. Besides statistical errors there are systematic error that are accounted by adding in quadrature a value of 0.04 magnitudes. Each source in each band is thus assigned with the error magnitude that was calculated using the following equation:
$$ magerr\_corrected = \sqrt{0.04^{2}+(\dfrac{1.0857}{SNR}\cdot K)^{2}} $$ SNR is FLUX$\_$ISO $ \//$ FLUXERR$\_$ISO and K is correction coefficient. SNR $>$ 5 cut was applied and the catalog at this stage consists of 26,585,000 sources. 

%For that reason $g' r' i' z'$ bands were convolved to the PSF of the one with the worst FWHM, i.e., u-band. In such a case the size variation in different bands will be attributed to the intrinsic difference in morphology at different wavelength. For the price of loosing some sources due to blending, more robust optical colors will be extracted. Matching is performed with $\tt IRAF/psfmatch$ task that requires supplying the image with the corresponding kernel - PSF matching function. Building a kernel requires  knowledge of the PSFs of the input image and a matched image. Construction of 27, 780 PSFs in Stripe 82 (5, 556 per band) is the most time-consuming part of the project.
%One of the widely-used ways to construct PSF is by using PSFEx software [Bertin, 2011]. We tested it with TPHOT and found that the quality of residual images is not satisfactory, so we decided to use more conservative algorithm in which IRAF/psf task creates a PSF function based on a set of selected point-like sources (so called PSF stars). Usually that strategy involves consecutive run of IRAF/find, IRAF/phot, IRAF/pstselect and IRAF/psf tasks that find stars, determine its magnitude, select the ones that do not have any morphological features and finally construct a PSF function. This algorithm, though very reliable, is also not ideal - in each image there are $\approx 10\%$ of sources that do not satisfy the criteria of PSF stars (are elongated, blended or have noisy background) and have to be rejected in the manual regime (selection in interactive mode). This is extremely time-consuming and inefficient given the total number of images in 5 bands. We decide to use a variation of this method to create PSF functions in a more automatic fashion.

%We run SExtractor on each SDSS image and only mark sources from the output catalog as a point-like, if their ”stellarity” index (determined by a CLASS STAR parameter) is close to 1, sources are bright, but not saturated, they do not lay within 2$\cdot$ PSF radius to the edge of the FITS image, and have all their flux contained within some small aperture. The later was calculated as the difference ”MAG diff” between MAG APER that returns all flux within some fixed aperture and MAG BEST that returns MAG AUTO value in the absence of contamination from the nearby sources and MAG ISOCOR otherwise. Exact parameters depend on the band and also were adjusted during test runs so that each SDSS image has no less than 20 PSF stars (stars were sorted by their magnitude, from bright to dim), with the maximum number of PSF stars limited to 160 (larger values significantly slow down IRAF).

\textit{I rewrote till this phrase}

%**matching all 5 bands to the PSF of the u-band**

%When all PSF stars are selected PSF was constructed using standard IRAF/psf. Then we run IRAF/seepsf, a task that takes the input PSF computed by the IRAF/psf task, consisting of the parameters of a 2D analytic function stored in the image header, and computes 21x21 pixel output PSF FITS image consisting of the sum of the analytic function and the residuals. Visual inspection of all PSF images revealed a certain fraction of defect PSFs. The reason for it can be either a blended source within the PSF radius, high background values due to the nearby bright star (within up to several arcmin), or noticeable elongation of the source (may be due to inclusion of a galaxy in our sample or due to intrinsic problems with SDSS image). The fraction of such images varied in different bands but generally was $\approx 4\%$ and for such images we re-run IRAF/psf in interactive mode (manually selecting PSF stars from a pre-selected catalog). All PSF images were normalized to unity total counts using wcstools/sumpix [Mink, 1998] and IRAF/imarith tasks. Several PSFs for all 5 bands and 6 different columns centered at RA=21:56:46 are shown in Figure 3.2. We used IRAF/lucy, a task that uses algorithm developed independently by Lucy [Lucy, 1974] and Richardson [Richardson, 1972], to create kernels for pairs of images in x- and u-bands, where ”x-” stands for g-, r-, i-, or z-band
 
%With kernels in hand it is now possible to run IRAF/psfmatch that convolves input image with the kernel to produce a psf matched output image (Figure 3.3). From when u-band, g-band, r-band etc images mean "SDSS Stripe 82 images stacked by J14 with new CRVAL and new CRPIX values", while g matched u, etc. means images in the g-band matched to the PSF of the u-band image. 
 
% *	catalog construction with SExtractor in dual mode, using r\_matched\_u band for detection
	%SDSS co-adds from J14 already have associated catalogs on the whole Stripe 82 region. We decide not to use it and create our own for several reasons:\\
%	-	J14 catalogs are not matched between the bands, resulting in different number of objects in different bands.\\
%	-	SNR at which the source is rejected is too high and there are a lot of sources that are identified in several bands when we perform our own photometry that were not included in J14 catalog.\\
%	-	Bright and saturated objects are detected as multiple sources, none of which is at the center of such object making correct SED fitting impossible. \\
%	On Figure~\ref{fig:catalog_comp} we show a comparison of the original catalog from J14 (sources within green regions) and our catalog (sources within red regions). In the next 2 subsections we show the strategy for catalog construction with consistent optical colors in all 5 SDSS bands.

%To construct a consistent optical catalog we run SExtractor in the dual mode - the first image is used for detection and astrometry information, while the second is used solely for photometry. We take r mathced u image as detection image and supply consequently all 5 bands to extract photometry of the sources and reject sources with SNR < 5 based on the r matched u flux. R matched u catalogs as well as corresponding segmentation maps also serve as an input to the TPHOT to derive consistent photometry on near-IR bands. Our parent catalog now consists of 26,585,000 sources. We notice that very few sources were rejected at this stage - objects that appear almost undetectable had FLUX AUTO / FLUXERR AUTO larger than 5 or even 7. 

% *	correction of the error magnitudes in g ,r ,i  and z matched u bands

%Photometry comparison between original x-band and x matched u-band catalogs shows a systematic underestimation of the errors associated with the source's flux (Figure 3.4). We believe that the reason for this was that we supplied to SExtractor science images matched to the u-band while the weight images were not matched to the u-band. This was the reason for high SNR ratios and low rejection rate for the faint sources. To correct for that we use STILTS code [Taylor, 2006] to match x-band and x matched u catalogs in 5 bands and for each separate image we calculate the mean ratio of the flux error before and after PSF matching. All magnitude errors in a given image are then multiplied by this coefficient. Figure 3.5 shows coefficients for all 5556 images for 4 bands matched to the u-band. We did not apply IRAF/psfmatch task to the u-band, but still calculated that correcting coefficient because r matched u FITS file was used as detection image in SExtractor. All coefficients for this band are less than 1. (CHECK THIS!)

%We anticipate that while corrected values account for statistical error, there should also be a systematic error in magnitude that needs to be taken care of. We performed a series of tests where different constant errors were added in quadrature to the reported magnitude error and then the goodness of fit on the graph z spec vs z phot was checked. We found that correlation is the tightest when 0.04 magnitude error is added in quadrature (so final error cannot be smaller than 0.04). Each source in each band was assigned by a error in magnitude that was calculated using the following equation:

%$$ magerr\_corrected = \sqrt{0.04^{2}+(\dfrac{1.0857}{SNR}\cdot correction\_coefficient)^{2}}$$

\subsection{preparatory work with input files to TPHOT} 
%	* SWarping of unWISE files
	
%We use SWarp to change the pixel scale of all unWISE images from original 2.75 to 2.772 to match the integer pixel scale ratio with SDSS images = 7. Now we construct unWISE PSF functions that is to be convolved with SDSS PSF to produce kernels - one of the most important set of the input files to TPHOT.	
*PSF construction

WISE Atlas Images are constructed by coadding many single-exposure images (IV.4.f). Because the number and relative orientation of single-exposures differ significantly between Atlas Tiles, and because the single-exposure PSFs vary with focal plane location (IV.4.c.iii), the PSF will be different for every Atlas Image and will vary with position on any given Atlas Image. ($http://wise2.ipac.caltech.edu/docs/release/allsky/expsup/sec4_4c.html$)


There are 240 unWISE images within Stripe 82 footprint per band. Bands w3 and w4
are too shallow and no reasonable flux can be extracted for the vast majority of
optical sources in these bands, so for the purpose of this project we only use w1
and w2 bands. We followed the strategy from the previous section (i.e., running SExtractor, selecting potential PSF stars using STILTS) to construct PSF for all 480 images with two major differences
in the procedure:
1) Center pixels of saturated sources in unWISE standard deviation images (unwise-
0000p000-w1-std-m.fits) always have zero value and that is an invalid input for TPHOT.
We use IRAF/imcalc to detect such pixels and change its value to ”9999”.
2) We construct each PSF function using IRAF/psf interactive mode. This was done to perform more robust selection of stars as 3 PSFs from one unWISE
frame are convolved with up to 72 SDSS PSFs and thus its quality is crucial - incorrect
PSF profile leads to wrong flux estimation associated with such sources and also
characteristic positive and negative ring-shaped patterns in the residuals.
\\
After running the IRAF/seepsf task we get PSF FITS images, 19x19 pixels each. All
unWISE PSFs are then normalized to unity total counts using IRAF/imarith and sub-sampled
in size by the factor of 7 using IRAF/imlintran (to 133x133 pixels) to match the pixel
scale ratio between unWISE and SDSS images. Result is presented on Figure 3.6	
	* kernel construction
We use r matched u band as a detection band for the optical sources and also as a high-resolution image for the template fitting. We run the same algorithm to create 5556 r matched u PSFs (not to be mixed up with the r matched u PSF that we use to change FWHM of the original r-band). IRAF/lucy is used to convolve normalized r matched u PSF with normalized and sub-sampled unWISE PSF to create kernels. These kernels serve as one of the inputs for TPHOT.

* naming convention

Both final and intermediate results are a combination of SDSS and unWISE data and
we introduce our new naming convention that is used throughout the project. E.g. file
kernel.w1.0000p015 12r u 111.fits is a kernel made by convolution of the PSF function
of the file unwise-0000p015-w1-img-m.fits and PSF function of the file S82 12r 111.fits
in r-band, that has been previously matched to the PSF of the u-band.


\section{Redshift and mass estimation}

The availability of near-IR filters helps improving the
z phot accuracy beyond z $\approx$ 1.3, where the 4000A break goes out
of the z 0 filter and the Lyman break is not yet detectable in the
u - band. (from 1809.03373.pdf)

\subsection{TPHOT} 
	used r matched u catalog and corresponding segmentation maps
	parameters, limitations, computation on the cluster 

\subsection{Final catalog with photometric redshifts} 
	matching by the source number
	comparison output fluxes from TPHOT to the unWISE catalogs

\subsection{EAZY} 
	redshift determination, matching to the spectroscopic redshifts
	comparison of the redshift with and without unWISE data (similar results)
** masking area around bright stars\\
** star-galaxy separation
** removing duplicate objects**

	
\section{Results}

\section{Discussion}


\section{Summary}


\acknowledgements


\bibliographystyle{plainnat}
%\bibliographystyle{unsrt}
\bibliography{ms}

%\input{table3_allpar.tex}

\newpage
\centerline{ {\bf APPENDIX}}
\appendix
\section{Prospective Automation of the Decomposition Process}

\section{Acknowledgments}
This publication makes use of data products from the Wide-field Infrared Survey Explorer, which is a joint project of the University of California, Los Angeles, and the Jet Propulsion Laboratory/California Institute of Technology, funded by the National Aeronautics and Space Administration.


\end{document}
